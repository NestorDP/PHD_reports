%% abtex2-modelo-trabalho-academico.tex, v-1.9.7 laurocesar
%% Copyright 2012-2018 by abnTeX2 group at http://www.abntex.net.br/ 
%%
%% This work may be distributed and/or modified under the
%% conditions of the LaTeX Project Public License, either version 1.3
%% of this license or (at your option) any later version.
%% The latest version of this license is in
%%   http://www.latex-project.org/lppl.txt
%% and version 1.3 or later is part of all distributions of LaTeX
%% version 2005/12/01 or later.
%%
%% This work has the LPPL maintenance status `maintained'.
%% 
%% The Current Maintainer of this work is the abnTeX2 team, led
%% by Lauro César Araujo. Further information are available on 
%% http://www.abntex.net.br/
%%
%% This work consists of the files abntex2-modelo-trabalho-academico.tex,
%% abntex2-modelo-include-comandos and abntex2-modelo-references.bib
%%

% ------------------------------------------------------------------------
% ------------------------------------------------------------------------
% abnTeX2: Modelo de Trabalho Academico (tese de doutorado, dissertacao de
% mestrado e trabalhos monograficos em geral) em conformidade com 
% ABNT NBR 14724:2011: Informacao e documentacao - Trabalhos academicos -
% Apresentacao
% ------------------------------------------------------------------------
% ------------------------------------------------------------------------

\documentclass[
	% -- opções da classe memoir --
	article,			% indica que é um artigo acadêmico
	12pt,						% tamanho da fonte
	oneside,					% para impressão em recto e verso. Oposto a oneside
	a4paper,					% tamanho do papel. 
	% -- opções da classe abntex2 --
	%chapter=TITLE,				% títulos de capítulos convertidos em letras maiúsculas
	%section=TITLE,				% títulos de seções convertidos em letras maiúsculas
	%subsection=TITLE,			% títulos de subseções convertidos em letras maiúsculas
	%subsubsection=TITLE,		% títulos de subsubseções convertidos em letras maiúsculas
	% -- opções do pacote babel --
	english,					% idioma adicional para hifenização
	french,						% idioma adicional para hifenização
	spanish,					% idioma adicional para hifenização
	brazil						% o último idioma é o principal do documento
	sumario=tradicional
	]{abntex2}


\usepackage{lmodern}			% Usa a fonte Latin Modern			
\usepackage[T1]{fontenc}		% Selecao de codigos de fonte.
\usepackage[utf8]{inputenc}		% Codificacao do documento (conversão automática dos acentos)
\usepackage{indentfirst}		% Indenta o primeiro parágrafo de cada seção.
\usepackage{color}				% Controle das cores
\usepackage{graphicx}			% Inclusão de gráficos
\usepackage{microtype} 			% para melhorias de justificação
\usepackage{lipsum}				% para geração de dummy text
\usepackage[brazilian,hyperpageref]{backref}	 % Paginas com as citações na bibl
\usepackage[alf]{abntex2cite}	% Citações padrão ABNT
\usepackage{microtype} 			% para melhorias de justificação
\usepackage{multirow}

\usepackage{pdfpages}
\usepackage{forest}
\usepackage{fontawesome}

\definecolor{thered}{rgb}{0.65,0.04,0.07}
\definecolor{thegreen}{rgb}{0.06,0.44,0.08}
\definecolor{thegrey}{gray}{0.5}
\definecolor{theshade}{rgb}{1,1,0.97}
\definecolor{theframe}{gray}{0.6}
\definecolor{blue}{RGB}{41,5,195} 

\usepackage{listings}
\lstset{
	basicstyle=\small, % print whole listing small
	keywordstyle=\color{blue}\bfseries, %\underbar,% underlined bold black keywords
	% keywords=[0]{uint32_t, exit},
	% keywordstyle=[0]\color{thered},
	identifierstyle=, % nothing happens
	commentstyle=\color{gray!85}, % white comments
	stringstyle=\color{thered}\ttfamily, % typewriter type for strings
	showstringspaces=false, % no special string spaces
	numbers=left,
	numberstyle=\tiny,
	stepnumber=2
}

\definecolor{folderbg}{RGB}{124,166,198}
\definecolor{folderborder}{RGB}{110,144,169}

\def\Size{4pt}
\tikzset{
  folder/.pic={
    \filldraw[draw=folderborder,top color=folderbg!50,bottom color=folderbg]
      (-1.05*\Size,0.2\Size+5pt) rectangle ++ (.75*\Size,-0.2\Size-5pt);  
    \filldraw[draw=folderborder,top color=folderbg!50,bottom color=folderbg]
      (-1.15*\Size,-\Size) rectangle (1.15*\Size,\Size);
  }
}


% Configurações do pacote backref
% Usado sem a opção hyperpageref de backref
\renewcommand{\backrefpagesname}{Citado na (s) página (s):~}
% Texto padrão antes do número das páginas
\renewcommand{\backref}{}
% Define os textos da citação
\renewcommand*{\backrefalt}[4]{
	\ifcase#1
		Nenhuma citação no texto.%
	\or%
		Citado na página #2.%
	\else
		Citado #1 vezes nas páginas #2.%
	\fi}%


%-------------------------------------------------------------------------------------
%		DADOS PESSOAIS
%		Informações básicas sobre os trabalho e os autores 
%-------------------------------------------------------------------------------------
\titulo{Relatório de atividades - Projeto de pesquisa I}
\autor{Nestor Dias Pereira Neto}
\local{Salvador}
\data{Julho 2023}
\orientador{Prof. Dr. Paulo César Machado de Abreu Farias}
\coorientador{Prof. Dr. Wagner Luiz Alves de Oliveira}


\instituicao{%
  Universidade Federal da Bahia - UFBA
  \par
  Escola Politécnica
  \par
  Programa de Pós-Graduação em Engenharia Elétrica}


%	Configurações de aparência do PDF final
% alterando o aspecto da cor azul
\definecolor{blue}{RGB}{41,5,195}

% informações do PDF
\makeatletter
\hypersetup{
     	%pagebackref=true,
		pdftitle={\@title}, 
		pdfauthor={\@author},
    	pdfsubject={\imprimirpreambulo},
	    pdfcreator={LaTeX with abnTeX2},
		pdfkeywords={abnt}{latex}{abntex}{abntex2}{trabalho acadêmico}, 
		colorlinks=true,       		% false: boxed links; true: colored links
    	linkcolor=blue,          	% color of internal links
    	citecolor=blue,        		% color of links to bibliography
    	filecolor=magenta,      		% color of file links
		urlcolor=blue,
		bookmarksdepth=4
}
\makeatother


% Posiciona figuras e tabelas no topo da página quando adicionadas sozinhas em um página em branco. 
% Ver https://github.com/abntex/abntex2/issues/170
\makeatletter
\setlength{\@fptop}{5pt} % Set distance from top of page to first float
\makeatother

% Possibilita criação de Quadros e Lista de quadros. Ver https://github.com/abntex/abntex2/issues/176
\newcommand{\quadroname}{Quadro}
\newcommand{\listofquadrosname}{Lista de quadros}

\newfloat[chapter]{quadro}{loq}{\quadroname}
\newlistof{listofquadros}{loq}{\listofquadrosname}
\newlistentry{quadro}{loq}{0}

% configurações para atender às regras da ABNT
\setfloatadjustment{quadro}{\centering}
\counterwithout{quadro}{chapter}
\renewcommand{\cftquadroname}{\quadroname\space} 
\renewcommand*{\cftquadroaftersnum}{\hfill--\hfill}

\setfloatlocations{quadro}{hbtp}

% O tamanho do parágrafo é dado por:
\setlength{\parindent}{1.3cm}

% Controle do espaçamento entre um parágrafo e outro:
\setlength{\parskip}{0.2cm}  % tente também \onelineskip


\makeindex


%%%%%%%%%%%%%%%%%%%%%%%%%%%%%%%%%%%%%%%%%%%%%%%%%%%%%%%%%%%%%%%%%%%%%%%%%%%%%%%%%%%%%%
%																					 %
% 							 Início do documento								     %
%																					 %
%%%%%%%%%%%%%%%%%%%%%%%%%%%%%%%%%%%%%%%%%%%%%%%%%%%%%%%%%%%%%%%%%%%%%%%%%%%%%%%%%%%%%%
\begin{document}


%\selectlanguage{english}
\selectlanguage{brazil}

% Retira espaço extra obsoleto entre as frases.
\frenchspacing

%-------------------------------------------------------------------------------------
% 		ELEMENTOS PRÉ-TEXTUAIS
%		Insire elementos pré textuais, ficha catalográfica 
%-------------------------------------------------------------------------------------
\pretextual

\imprimircapa


% inserir o sumario
\pdfbookmark[0]{\contentsname}{toc}
\tableofcontents*
\cleardoublepage%

%-------------------------------------------------------------------------------------
% 		ELEMENTOS TEXTUAIS
%		Informações relevantes ao trabalho, capítulos e suas respectivas seções.
%-------------------------------------------------------------------------------------
\textual%

\chapter{Introdução}\label{cap:intro}


\begin{itemize}
    \item \textbf{Como estabelecer a comunicação entre o ROS e um sistema de processamento auxiliar 
embarcado em um FPGA?}

\end{itemize}

\lipsum[4]


% \section{Objetivos}

% \subsection{Objetivo Geral}

% Desenvolver uma solução para estabelecer comunicação entre \textit{Field-Programmable Gate Array - FPGA}, 
% configurado como um co-processador de vídeo.

% \subsection{Objetivos Específicos}

% \begin{itemize}
%     \item Estudar teoria dos assuntos relevantes ao projeto: Verilog HDL, Embedded Linx,  Cyclone V, 
%     TCP/IP Stack, ROS\@;
%     \item Estudar conceitos de programação de redes usando sockets em liguagem C++ e detalhes dos protocolos da rede TCP/IP usada para comunicação interna dos nós e serviços ROS\@;
%     \item Implementar distribuição Embedded Linux para processador ARM embarcado no SoC Cyclone V da Intel;
%     \item Estabelecer comunicação entre o ROS e o Cyclone V, através da tecnologia Gigabit Ethernet;
%     \item Desenvolver aplicação em Verilog para testar comunicação;
%     \item Avaliar o desempenho da rede entre o computador e o protótipo após a inclusão do FPGA ao sistema.
% \end{itemize}



% 

\lipsum[4]
% \section{Atividades de pesquisa}\label{cap:pesquisa}

\lipsum[4]
% \chapter{Atividades de ensino}\label{cap:ensino}

\lipsum[4]
% \section{Trabalhos e Publicações}\label{cap:pub}

\lipsum[4]
% \section{Conclusão}\label{cap:conclusao}

\lipsum[4]
% \section{Futuras direções}\label{cap:futuro}

\lipsum[4]

% Adiciona espaço de parte no Sumário
\phantompart%
%-------------------------------------------------------------------------------------
%		CONTEÚDO PÓS-TEXTUAL
%		Parte do documento que aparece após o capítulo de conclusão, incluindo 
%		bibliografia e anexos.
%-------------------------------------------------------------------------------------
\postextual\bibliography{references} 

%\glossary

% Apendices
%-------------------------------
% \begin{apendicesenv}
% \partapendices
% % ----------------------------------------------------------
\section{Quisque libero justo}
% ----------------------------------------------------------

\lipsum[50]

% ----------------------------------------------------------
\section{Nullam elementum urna vel imperdiet sodales elit ipsum pharetra ligula
ac pretium ante justo a nulla curabitur tristique arcu eu metus}
% ----------------------------------------------------------
\lipsum[55-57]
% \end{apendicesenv}

% Anexos
%-------------------------------
% \begin{anexosenv}
% \partanexos
% % ---
\chapter{Morbi ultrices rutrum lorem.}
% ---
\lipsum[30]
% % ---
\section{Cras non urna sed feugiat cum sociis natoque penatibus et magnis dis
parturient montes nascetur ridiculus mus}
% ---

\lipsum[31]
% 
% ---
\chapter{Fusce facilisis lacinia dui}
% ---

\lipsum[32]
% \end{anexosenv}

% Indice remissivo
%-------------------------------
% \phantompart
% \printindex

\end{document}
