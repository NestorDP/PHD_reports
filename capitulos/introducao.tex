\section{Introdução}\label{cap:intro}

O Grande Colisor de Hádrons (LHC, na sigla em inglês) é o maior e mais poderoso acelerador de partículas do mundo. Localizado no Centro Europeu de Pesquisa Nuclear (CERN), na fronteira entre a França e a Suíça, o LHC tem como objetivo investigar as propriedades fundamentais da matéria e as leis que governam o universo em suas escalas mais microscópicas.

Inaugurado em 2008, o LHC é um anel de aceleração circular com cerca de 27 quilômetros de circunferência. Nele, partículas subatômicas, como prótons, são aceleradas a velocidades próximas à velocidade da luz e colidem entre si em pontos específicos ao longo do anel. Essas colisões de altíssima energia permitem que os cientistas observem fenômenos e partículas que não podem ser estudados em condições normais.

Dentre os experimentos realizados no LHC, destaca-se o Experimento ATLAS (A Toroidal LHC ApparatuS). O ATLAS é um dos quatro principais detectores de partículas do LHC e tem como objetivo investigar uma ampla gama de fenômenos físicos que ocorrem nas colisões de partículas de alta energia.

O ATLAS é composto por uma série de detectores de alta precisão posicionados em diferentes camadas ao redor do ponto de colisão. Esses detectores permitem a identificação e o estudo de diferentes partículas resultantes das colisões, fornecendo informações cruciais para a compreensão das propriedades fundamentais da matéria.

Durante este semestre, dediquei meu tempo para estudar e me familiarizar com as informações relacionadas ao LHC e ao experimento ATLAS. Essa etapa foi essencial para compreender os termos técnicos e a estrutura desse experimento científico impressionante. Este contato inicial contribuiu para estabelecer uma base de conhecimento para a continuidade dos estudos e das pesquisas durante o desenvolvimento da minha tese.


\section{Metodologia}\label{cap:metodo}

Nesta seção, será descrita a metodologia utilizada para conduzir as atividades acadêmicas no âmbito deste relatório. As principais etapas e abordagens adotadas serão apresentadas, abrangendo: estudos teóricos em documentos, manuais e reuniões técnicas
 
Para embasar a pesquisa e o desenvolvimento futuro da dissertação, foram realizados estudos teóricos em documentos e manuais desenvolvidos pelas equipes do CERN. Artigos científicos foram revisados para obter uma base de conhecimento teórico sobre o estado da arte a respeito das técnicas utilizadas na fase do trigger do experimento ATLAS. Esses estudos incluíram a análise crítica de artigos científicos, livros, teses e outras publicações acadêmicas pertinentes. Os materiais foram selecionados com base em sua relevância, autoridade e atualidade, a fim de sustentar a fundamentação teórica do trabalho.

Reuniões técnicas desempenharam um papel importante, contribuindo com a familiarização com o experimento. Durante essas reuniões, foram discutidos tópicos específicos relacionados às pesquisas mais recentes sobre o ATLAS, além de expor questões como os desafios encontrados pelos pesquisadores e alguns resultados preliminares. Essas reuniões envolveram a interação com orientadores, colaboradores de pesquisa e outros especialistas da área. A troca de ideias e conhecimentos nessas ocasiões contribuiu significativamente para o aprimoramento da base de conhecimento a respeito do ATLAS, além de trazer novas direções para o andamento da pesquisa.

É importante ressaltar que essas atividades foram complementares e fundamentais para aprimorar a pesquisa e estimular a troca de conhecimentos com outros pesquisadores da área. Ao adotar essa metodologia, foi possível obter uma visão ampla e atualizada do campo de estudo, além de embasar a pesquisa em evidências científicas sólidas e informações relevantes provenientes de fontes confiáveis.


\section{Atividades de pesquisa}\label{cap:pesquisa}
Neste capítulo, serão apresentadas algumas das atividades de pesquisa realizadas neste semestre. Durante o período em questão, concentrei meus principais estudos teóricos nos documentos disponibilizados pelo CERN. Dentre esses documentos, destaco alguns que possuem grande relevância para o desenvolvimento da minha pesquisa. Esses artigos foram selecionados por sua relevância para a compreensão do sistema de gatilho (trigger) e aquisição de dados (data acquisition) do Experimento ATLAS no Grande Colisor de Hádrons (LHC) do CERN.


\textit{\textbf{Operation of the ATLAS Trigger System in Run 2:}} O artigo Operation of the ATLAS Trigger System in Run 2" fornece uma descrição detalhada do funcionamento do sistema de gatilho do Experimento ATLAS durante a segunda fase de operações do LHC. Com base nesse artigo, foram realizados estudos para compreender o desempenho do sistema de gatilho, sua eficiência na seleção de eventos relevantes e sua capacidade de gerenciar o alto volume de dados produzidos nas colisões de partículas de alta energia. A análise desses estudos permitiu identificar possíveis melhorias no sistema de gatilho e aquisição de dados, além de fornecer informações valiosas para o aprimoramento contínuo do experimento \cite{The_ATLAS_collaboration_2020}.

\textit{\textbf{The ATLAS Experiment at the CERN Large Hadron Collider:}} O artigo \textit{The ATLAS Experiment at the CERN Large Hadron Collider} apresenta uma visão geral abrangente do Experimento ATLAS, descrevendo sua estrutura, componentes e objetivos científicos. Com base nesse artigo, foi possível compreender melhor o contexto em que os estudos foram realizados e a importância do sistema de gatilho e aquisição de dados para o sucesso do experimento. Além disso, a análise desse artigo permitiu identificar os principais desafios enfrentados pelo ATLAS e as soluções adotadas para superá-los, fornecendo uma base sólida para a compreensão dos estudos realizados \cite{The_ATLAS_Collaboration_2008}.

\textit{\textbf{Technical Design Report for the Phase-II Upgrade of the ATLAS Trigger  and Data Acquisition System:}} O artigo \textit{Technical Design Report for the Phase-II Upgrade of the ATLAS Trigger and Data Acquisition System} apresenta o relatório técnico para a atualização da fase II do sistema de gatilho e aquisição de dados do ATLAS. Com base nesse artigo, foram conduzidos estudos para compreender as melhorias planejadas para o sistema de gatilho e aquisição de dados, avaliando seus impactos no desempenho geral do experimento. A análise desse relatório técnico permitiu obter informações valiosas sobre as inovações propostas, os desafios técnicos envolvidos e as perspectivas futuras para o aprimoramento do ATLAS \cite{CERN_LHCC_2017}.

\textit{\textbf{ATLAS TDAQ Phase-II Upgrade: System Specifications for the Global Trigger:}} Esse artigo detalha as especificações do Sistema de Disparo Global do ATLAS para a Fase II do aprimoramento do TDAQ (Trigger and Data Acquisition System). O estudo baseado neste artigo se concentrou nas especificações técnicas do Sistema de Disparo Global, considerando os requisitos de filtragem e seleção de eventos em um ambiente de alta taxa de colisões. Foram analisados os desafios enfrentados e as soluções propostas para garantir a eficiência e o desempenho do Sistema de Disparo Global durante a Fase II do upgrade \cite{CERN_LHC_2022}.

\textit{\textbf{ATLAS Global Event Processor Phase-II Upgrade: Trigger Firmware Algorithm: Topoclustering:}} Esse artigo aborda o algoritmo de firmware de disparo \textit{Topoclustering} usado na Fase II do upgrade do Processador Global de Eventos do ATLAS. O estudo baseado neste artigo concentrou-se na compreensão do algoritmo, sua implementação e impacto no desempenho geral do Sistema de Disparo. Foram analisados os critérios de seleção de eventos usando o algoritmo de \textit{Topoclustering} e aprimoramentos planejados para otimizar a eficiência de filtragem e seleção de eventos de interesse científico \cite{CERN_LHC_2023}.

No contexto das minhas atividades acadêmicas, além do estudo teórico mencionado anteriormente, iniciei a especialização do Programa de cursos integrados Deep Learning da deeplearning.ai. Recentemente, concluí a primeira etapa dos cinco cursos necessários para a conclusão dessa especialização. O objetivo dessa formação é aprofundar o entendimento sobre os fundamentos, desafios e consequências do aprendizado profundo, fornecendo conhecimentos teóricos e práticos para explorar e aplicar essa área em diferentes domínios. Os cursos são interconectados, cada um abordando aspectos específicos do aprendizado profundo. Essa especialização foi indicada pelos meus orientadores como uma fonte relevante de conhecimento, que certamente contribuirá para as próximas etapas do desenvolvimento da minha pesquisa.


\section{Atividades de ensino}\label{cap:ensino}

No último semestre, não foram realizadas atividades de ensino.

\section{Publicações}\label{cap:pub}

No último semestre, não foram realizadas publicações.

\section{Conclusão}\label{cap:conclusao}

Os estudos realizados com base nos documentos mencionados proporcionaram uma compreensão aprofundada do sistema de gatilho e aquisição de dados do experimento ATLAS no LHC do CERN. A análise desses documentos contribuiu para a identificação de melhorias potenciais no sistema, bem como para a compreensão dos desafios técnicos envolvidos em sua atualização e aprimoramento contínuos.

Até o momento, concluí o primeiro curso do Programa de cursos integrados Deep Learning da deeplearning.ai. Por meio deste curso, estou tendo a oportunidade de aprofundar meu conhecimento em deep learning, compreender técnicas avançadas e explorar suas aplicações. Essa formação será fundamental para fortalecer meu embasamento teórico e prático, permitindo que eu aplique esses conhecimentos no meu projeto de pesquisa.


\section{Futuras direções}\label{cap:futuro}

Para as atividades futuras, a pesquisa concentrará esforços na análise e implementação dos algoritmos do sistema de trigger do ATLAS, com foco na sua implementação em FPGA (\textit{Field-Programmable Gate Array}). O objetivo é investigar a eficiência e o desempenho desses algoritmos quando executados em hardware reconfigurável. Serão realizados estudos sobre a arquitetura dos algoritmos e as estratégias de implementação em FPGA, buscando otimizar os recursos de processamento e garantir a eficiência do sistema de trigger do ATLAS.

Além disso, a especialização no Programa de cursos integrados Deep Learning da deeplearning.ai será continuada. Com a intenção de concluir os demais cursos, aprofundarei meu conhecimento em \textit{deep learning} e explorarei técnicas avançadas para aplicá-las em pesquisas futuras e projetos relacionados à minha tese.

A combinação dessas atividades, a análise dos algoritmos do sistema de trigger do ATLAS implementados em FPGA e a continuidade na especialização em aprendizado profundo, permitirá a continuidade do desenvolvimento da pesquisa. Através dessas atividades futuras, pretende-se contribuir para melhorar continuamente o sistema de trigger do ATLAS, além de explorar o potencial do aprendizado profundo para avançar na compreensão da física de partículas.








% \begin{itemize}
%     \item \textbf{Como estabelecer a comunicação entre o ROS e um sistema de processamento auxiliar 
% embarcado em um FPGA?}

% \end{itemize}

% \lipsum[4]


% \section{Objetivos}
% \lipsum[4]

% \subsection{Objetivo Geral}

% Desenvolver uma solução para estabelecer comunicação entre \textit{Field-Programmable Gate Array - FPGA}, 
% configurado como um co-processador de vídeo.

% \subsection{Objetivos Específicos}

% \begin{itemize}
%     \item Estudar teoria dos assuntos relevantes ao projeto: Verilog HDL, Embedded Linx,  Cyclone V, 
%     TCP/IP Stack, ROS\@;
%     \item Estudar conceitos de programação de redes usando sockets em liguagem C++ e detalhes dos protocolos da rede TCP/IP usada para comunicação interna dos nós e serviços ROS\@;
%     \item Implementar distribuição Embedded Linux para processador ARM embarcado no SoC Cyclone V da Intel;
%     \item Estabelecer comunicação entre o ROS e o Cyclone V, através da tecnologia Gigabit Ethernet;
%     \item Desenvolver aplicação em Verilog para testar comunicação;
%     \item Avaliar o desempenho da rede entre o computador e o protótipo após a inclusão do FPGA ao sistema.
% \end{itemize}


