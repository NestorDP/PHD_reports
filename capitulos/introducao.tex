\section{Introdução}\label{cap:intro}

O Grande Colisor de Hádrons (LHC, na sigla em inglês) é o maior e mais poderoso acelerador de partículas do mundo. Localizado no Centro Europeu de Pesquisa Nuclear (CERN), na fronteira entre a França e a Suíça, o LHC tem como objetivo investigar as propriedades fundamentais da matéria e as leis que governam o universo em suas escalas mais microscópicas.

Inaugurado em 2008, o LHC é um anel de aceleração circular com cerca de 27 quilômetros de circunferência. Nele, partículas subatômicas, como prótons ou íons pesados, são aceleradas a velocidades próximas à velocidade da luz e colidem entre si em pontos específicos ao longo do anel. Essas colisões de altíssima energia permitem que os cientistas observem fenômenos e partículas que não podem ser estudados em condições normais.

Dentre os experimentos realizados no LHC, destaca-se o Experimento ATLAS (A Toroidal LHC ApparatuS). O ATLAS é um dos quatro principais detectores de partículas do LHC e tem como objetivo investigar uma ampla gama de fenômenos físicos que ocorrem nas colisões de partículas de alta energia.

O ATLAS é composto por uma série de detectores de alta precisão posicionados em diferentes camadas ao redor do ponto de colisão. Esses detectores permitem a identificação e o estudo de diferentes partículas resultantes das colisões, fornecendo informações cruciais para a compreensão das propriedades fundamentais da matéria.

Um dos principais objetivos do ATLAS é a busca pelo famoso bóson de Higgs, uma partícula hipotética que é fundamental para o modelo padrão da física de partículas e desempenha um papel crucial na explicação da origem da massa das partículas. Em 2012, o ATLAS e o experimento CMS (Compact Muon Solenoid) anunciaram conjuntamente a descoberta do bóson de Higgs, um marco histórico que corroborou a validade do modelo padrão e rendeu o Prêmio Nobel de Física em 2013 a François Englert e Peter Higgs.

Além da busca pelo bóson de Higgs, o ATLAS também está envolvido em pesquisas que abrangem áreas como a física de partículas além do modelo padrão, a busca por partículas exóticas, a compreensão da matéria escura e a investigação da assimetria entre matéria e antimatéria no universo.

Em suma, o LHC do CERN e o Experimento ATLAS são empreendimentos científicos de magnitude impressionante, que permitem a exploração das fronteiras do conhecimento da física de partículas. Através desses esforços, cientistas de todo o mundo buscam avançar nosso entendimento sobre as leis fundamentais da natureza, promovendo descobertas e insights que podem ter implicações profundas para nossa compreensão do universo.

8888888888888 88888888888 88888888 88888
888 8888888 88888888 88888888 88888888 8888 888 888888 88
888 8888 8888 8888888 8888888888 888
INCLUIR INFORMAÇÕES SOBRE A ATUALIZAÇÃO DO EXPERIMENTO ATLAS 8888888 8 8888888 888 888888 88 88888 88 88888888 888 88888 88888 8888 88888 88 8888888 

\section{Metodologia}\label{cap:metodo}

Nesta seção, descreveremos a metodologia utilizada para conduzir as atividades acadêmicas no âmbito deste relatório. As principais etapas e abordagens adotadas serão apresentadas, abrangendo estudos teóricos em documentos e manuais, reuniões técnicas e a participação em seminários como ouvinte.

Estudos Teóricos em Documentos e Manuais:
Para embasar a pesquisa e a produção acadêmica, foram realizados estudos teóricos extensivos em documentos e manuais relevantes na área de estudo. A literatura científica especializada foi revisada para obter uma base sólida de conhecimento teórico sobre o tema de pesquisa. Esses estudos incluíram a análise crítica de artigos científicos, livros, teses e outras publicações acadêmicas pertinentes. Os materiais foram selecionados com base em sua relevância, autoridade e atualidade, a fim de sustentar a fundamentação teórica do trabalho.

Reuniões Técnicas:
Reuniões técnicas desempenharam um papel importante no desenvolvimento do trabalho. Durante essas reuniões, foram discutidos tópicos específicos relacionados à pesquisa em andamento, além de questões metodológicas, desafios encontrados e resultados preliminares. Essas reuniões envolveram a interação com orientadores, colaboradores de pesquisa e outros especialistas da área. A troca de ideias e conhecimentos nessas ocasiões contribuiu significativamente para o aprimoramento da pesquisa e a definição de novas direções.

Participação em Seminários como Ouvinte:
Como parte do desenvolvimento acadêmico e da atualização constante, participou-se ativamente de seminários científicos e acadêmicos relevantes à área de pesquisa. A participação ocorreu principalmente como ouvinte, possibilitando o acesso a pesquisas recentes, ideias inovadoras e tendências emergentes na área. Durante os seminários, foram acompanhadas apresentações de renomados especialistas, o que proporcionou um ambiente propício à reflexão crítica e à ampliação do conhecimento em questões relacionadas ao tema de estudo.

Ao combinar essas abordagens metodológicas, buscou-se estabelecer uma base sólida de conhecimento teórico e técnico para embasar a pesquisa e a produção acadêmica. A revisão da literatura científica, a participação em reuniões técnicas e a presença em seminários como ouvinte contribuíram para a compreensão aprofundada do campo de estudo, a identificação de lacunas de conhecimento e o fortalecimento da fundamentação teórica do trabalho realizado.

É importante ressaltar que essas atividades foram complementares e fundamentais para aprimorar a pesquisa e estimular a troca de conhecimentos com outros pesquisadores da área. Ao adotar essa metodologia, foi possível obter uma visão ampla e atualizada do campo de estudo, além de embasar a pesquisa em evidências científicas sólidas e informações relevantes provenientes de fontes confiáveis.

\section{Atividades de pesquisa}\label{cap:pesquisa}

\lipsum[4]

\section{Atividades de ensino}\label{cap:ensino}

\lipsum[4]

\section{Trabalhos e Publicações}\label{cap:pub}

\lipsum[4]

\section{Conclusão}\label{cap:conclusao}

\lipsum[4]

\section{Futuras direções}\label{cap:futuro}

\lipsum[4]








% \begin{itemize}
%     \item \textbf{Como estabelecer a comunicação entre o ROS e um sistema de processamento auxiliar 
% embarcado em um FPGA?}

% \end{itemize}

% \lipsum[4]


% \section{Objetivos}
% \lipsum[4]

% \subsection{Objetivo Geral}

% Desenvolver uma solução para estabelecer comunicação entre \textit{Field-Programmable Gate Array - FPGA}, 
% configurado como um co-processador de vídeo.

% \subsection{Objetivos Específicos}

% \begin{itemize}
%     \item Estudar teoria dos assuntos relevantes ao projeto: Verilog HDL, Embedded Linx,  Cyclone V, 
%     TCP/IP Stack, ROS\@;
%     \item Estudar conceitos de programação de redes usando sockets em liguagem C++ e detalhes dos protocolos da rede TCP/IP usada para comunicação interna dos nós e serviços ROS\@;
%     \item Implementar distribuição Embedded Linux para processador ARM embarcado no SoC Cyclone V da Intel;
%     \item Estabelecer comunicação entre o ROS e o Cyclone V, através da tecnologia Gigabit Ethernet;
%     \item Desenvolver aplicação em Verilog para testar comunicação;
%     \item Avaliar o desempenho da rede entre o computador e o protótipo após a inclusão do FPGA ao sistema.
% \end{itemize}


