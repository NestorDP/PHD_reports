\chapter{Introdução}\label{cap:intro}


\begin{itemize}
    \item \textbf{Como estabelecer a comunicação entre o ROS e um sistema de processamento auxiliar 
embarcado em um FPGA?}

\end{itemize}

\lipsum[4]


% \section{Objetivos}

% \subsection{Objetivo Geral}

% Desenvolver uma solução para estabelecer comunicação entre \textit{Field-Programmable Gate Array - FPGA}, 
% configurado como um co-processador de vídeo.

% \subsection{Objetivos Específicos}

% \begin{itemize}
%     \item Estudar teoria dos assuntos relevantes ao projeto: Verilog HDL, Embedded Linx,  Cyclone V, 
%     TCP/IP Stack, ROS\@;
%     \item Estudar conceitos de programação de redes usando sockets em liguagem C++ e detalhes dos protocolos da rede TCP/IP usada para comunicação interna dos nós e serviços ROS\@;
%     \item Implementar distribuição Embedded Linux para processador ARM embarcado no SoC Cyclone V da Intel;
%     \item Estabelecer comunicação entre o ROS e o Cyclone V, através da tecnologia Gigabit Ethernet;
%     \item Desenvolver aplicação em Verilog para testar comunicação;
%     \item Avaliar o desempenho da rede entre o computador e o protótipo após a inclusão do FPGA ao sistema.
% \end{itemize}


